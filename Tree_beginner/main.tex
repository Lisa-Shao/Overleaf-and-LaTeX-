\documentclass[a4paper,12pt]{article}
\usepackage{amsmath}
\usepackage{booktabs}
\usepackage{geometry}
\usepackage{graphicx} % Required for inserting images
\usepackage[linguistics]{forest}
\usepackage{linguex}
\usepackage{listings}
\usepackage{tikz}
\usepackage{qtree}
\usepackage{setspace}
\usepackage{url}
\usepackage[normalem]{ulem}
\geometry{
 a4paper,
 total={170mm,257mm},
 }

\title{A Quickstart Guide for Drawing Tree Diagrams in \LaTeX}
\author{\textbf{Lisa Shao}}
\date{October 21, 2025}

\begin{document}
\maketitle

\begin{abstract}
This guide aims to provide a beginner-friendly introduction to using the \texttt{forest} and \texttt{qtree} packages in \LaTeX{} to draw clean tree diagrams for your linguistic assignments. ;-)
\end{abstract}

\tableofcontents

\begin{spacing}{1.5}
\section{The \LaTeX\ Basics: Preamble and Packages}
\LaTeX\ (pronounced /l'tek/) is powerful because of its \textbf{Preamble}, which defines the document type and loads special packages that give you extra features. Using \LaTeX\ ensures your work is \textbf{automatic, professional, and consistent}.

The preamble is where you tell the compiler what tools you'll be using. Since we're drawing trees, we need these packages:
\begin{center}
 \noindent\texttt{\string\usepackage[linguistics]\{forest\}} \\
% This is the tree-drawing package
\texttt{\string\usepackage\{amsmath, amssymb\}} \\ % For math symbols and semantic types
\texttt{\string\usepackage\{tikz\}}        % The original    
\end{center}

Always put this line in your preamble to tell \LaTeX\ you'll be using this tool.

\textbf{A Note } The \texttt{forest} package is actually built on top of the incredibly powerful \textbf{TikZ} package. We won't study TikZ directly because its manual is over 1,000 pages, but \texttt{forest} gives us an easy, user-friendly way specifically for drawing trees.

\section{Drawing Trees with the \texttt{forest} Package}

The \texttt{forest} package uses a simple bracket notation to represent the hierarchy of a tree.

\subsection{Environment Setting}

Before any tree coding starts, you must create the \textbf{environment} where the code will run. \textbf{NEVER} forget these lines:

\begin{center}
\verb|\begin{forest}| \\
\verb|type your codes here...|\\
\verb|\end{forest}|
\end{center}



\noindent \textbf{Note:} If you use an editor like VS Code, type \texttt{\textbackslash begin\{forest\}}, and the closing command \texttt{\textbackslash end\{forest\}} will often appear automatically!

\subsection{Simple Tree Structure}

Each set of brackets \texttt{[]} represents a single \textbf{node} in the tree. The first item inside the brackets is the node's label (the \textbf{mother}), and anything else inside represents its \textbf{daughters}.

\subsubsection*{Example 1: A Single Branch}
The code \texttt{[N [Kitty]]} produces:

\ex. \begin{forest}
[N [Kitty]]
\end{forest}

\subsubsection*{Example 2: Two Branches (\textit{Kitty meows})}
To draw a branching structure, you simply place the daughter nodes next to each other, separated by a space, inside the mother node's brackets.

The sentence \textit{Kitty meows} has a structure like: S $\rightarrow$ DP+VP.

We can firstly start with the POS (part of speech) labeling since it is unary in this case.
\begin{center}
\noindent\verb|\begin{forest}| \\
\verb|[ProperN[Kitty]]| \\
\verb|[V_{itr} [Meows]]| \\
\verb|\end{forest}| \\   
\end{center}


Then we can do the syntactic analysis, adding \textbf{S, DP and VP}.

\begin{center}
\verb|\begin{forest}| \\
\verb|[S| \\
\verb|    [DP [ProperN [Kitty]]]|\\
\verb|    [VP [V_{itr} [meows]]]|\\
\verb|]| \\
\verb|\end{forest}|
\end{center}


\noindent This produces:

\ex. 
\begin{forest}
[S
    [DP [ProperN [Kitty]]]
    [VP [V_{itr} [meows]]]
]
\end{forest}

\subsection{Adjusting node spacing}
Although \texttt{forest} will arrange the nodes in the tree for you, you can still adjust both the horizontal and the vertical spacing, and the empty space around the nodes. 

The default setting is
\begin{center}
\verb|\forestset{default preamble={for tree={s sep=10mm, inner sep=0, l=0}}}|   
\end{center}


\noindent \textbf{Something to note}:
\begin{itemize}
    \item The horizontal spacing is controlled by the \texttt{s sep} command, the vertical (or level) spacing by the \texttt{l} command. 
    \item  The \texttt{inner sep} command controls the empty space around the nodes.
\end{itemize}

You can specify absolute values for these parameters, as in the example
below, or increase or decrease their default values as calculated by forest.

This is done either by multiplication (e.g. l*=3 multiplies the default level distance by 3), or by addition or subtraction (e.g. l+=3mm adds 3mm to the default level distance, l-=3mm subtracts 3mm).




\subsection{Complex Tree Structure}
Let's analyze the transitive sentence \textit{Andy likes Billy} with a complete phrase structure.
\begin{enumerate}
    \item Start by encoding each word with its POS label: \texttt{[ProperN [Andy]]}, \texttt{[V [likes]]}, \texttt{[ProperN [Billy]]}.
    \item Group them into phrases using brackets: $\rightarrow$ \texttt{[NP [ProperN [Andy]]} and \texttt{[NP [ProperN [Billy]]}.
    \item Finally, link everything up to the main S node.
\end{enumerate}

\subsubsection*{Example 3: A whole sentence (\textit{Andy likes Billy})}
The complete code looks like this (Notice the use of V $'$ for V-bar, and that we must always ensure every bracket is closed!)

\begin{verbatim}
\begin{forest}
[S
    [NP
        [ProperN [Andy]]
    ]
    [VP
        [V$'$
            [V [likes]]
            [NP 
                [ProperN [Billy]]
            ]
        ]
    ]
]
\end{forest}
\end{verbatim}

\noindent And here is the output:

\ex.
\begin{forest}
[S
    [NP
        [ProperN [Andy]]
    ]
    [VP
        [V$'$
            [V [likes]]
            [NP
                [ProperN [Billy]]
            ]
        ]
    ]
]
\end{forest}

Note: in syntax, there is triangle to represent a whole phrase.
We use the \texttt{root} command, just type \texttt{[...,roof]} after the node.

\subsubsection*{Example 4: Roof (\textit{Andy likes little Billy})}
\begin{verbatim}
\begin{forest}
[S
    [NP
        [ProperN [Andy]]
    ]
    [VP
        [V$'$
            [V [likes]]
            [NP [little Billy, roof]
            ]
        ]
    ]
]
\end{forest}
\end{verbatim}

Then we can get:
\ex.
\begin{forest}
[S
    [NP
        [ProperN [Andy]]
    ]
    [VP
        [V$'$
            [V [likes]]
            [NP [little Billy, roof]
            ]
        ]
    ]
]
\end{forest}

\noindent In the next section, we'll look at how to add types like $\langle e \rangle$ and $\langle t \rangle$ for semantics or $\phi$-features for formal syntax

\subsection{Features}
In formal Semantics (this course), a tree should be drawn like this:
\begin{figure}[h]
    \centering
    \begin{forest}
    for tree = {s sep = 10mm , inner sep = 0pt, l = 0pt}
[DP
    %[D \\[{$\langle \langle e,t \rangle,e \rangle$ \\ $\lambda P.\, \iota x.\text{P}(x)$}[the]]
    [D \\{$\langle \langle e,t \rangle,e \rangle$ \\ $\lambda P.\, \iota x.\text{P}(x)$}
    [the]]
    [N$'$ \\ {$\langle e,t \rangle$ \\ $\lambda x.\,[\text{Textbook}(x) \land \text{On}(x,\text{sem})]$}
        [N \\{$\langle e,t\rangle$ \\ $\lambda x.\,\text{Textbook}(x)$} [textbook]]
        [PP \\{$\langle e,t \rangle$ \\ $\lambda x. \text{On}(x,\text{sem})$}
            [P \\{$\langle e, \langle e,t \rangle \rangle$ \\ $\lambda y \lambda x. \text{On}(x,y)$} [on]]
            [DP \\ e \\ sem [semantics]]
    ]
    ]

]
    \end{forest}
    \caption{Adapted from Heim and Kratzer (1998, p.~5)}
\end{figure}

For semantic types, we use \{$\langle ...\rangle$\} after part of speech and before the word.
Here is an example:
\subsubsection*{Example 5: semantic types(\textit{Everything vanished})}

We can type like this:
\begin{verbatim}
\begin{forest}
for tree = { s sep = 10mm , inner sep = 0pt, l = 0pt}
[S \\ t
    [DP\\{$\langle \langle e,t \rangle, t\rangle$} [everything]]
    [VP\\{$\langle e,t \rangle$}[V\\{$\langle e,t \rangle$}[vanised]]]
]
\end{forest}   
\end{verbatim}

Then the first semantic tree you have drawn in your life is:
\ex.
\begin{forest}
for tree = { s sep = 10mm , inner sep = 0pt, l = 0pt}
[S \\ t
    [DP\\{$\langle \langle e,t \rangle, t\rangle$} [everything]]
    [VP\\{$\langle e,t \rangle$}[V\\{$\langle e,t \rangle$}[vanised]]]
]
\end{forest}




Adding a feature needs to be done like {[feature]}, which is to put curly brackets outside square brackets. 

\textbf{Basic Feature:} Put the feature directly after the node's label, but before the children.

We use line breaks (\texttt{\textbackslash\textbackslash}) and custom spacing (\texttt{s sep}) to achieve this.

\subsubsection*{Example 6: $\phi$-feature(\textit{Marie Curie})}
\ex.
\begin{forest}
    [ProperN[Marie Curie {[u$\phi$: 3rd, sg, fem]}]]
\end{forest}


\subsection {Movements}
We need to draw arrows for movements, and this requires two steps: 
\begin{enumerate}
    \item \textbf{Naming Nodes:} Inside a node's square brackets, use the option \texttt{,name=LABEL}. For movement, you usually name the source (\texttt{src}) and the target (\texttt{tgt}).
    \item \textbf{Drawing the Arrow:} The \texttt{\textbackslash draw} command is placed *after* the entire \texttt{forest} environment. \textbf{Here we are using a \texttt{TikZ} command. So the package \texttt{TikZ} should be added}
\end{enumerate}

\subsubsection*{Example 7: movement(\textit{A simple Wh-Movement example})}
\ex. \textbf{Wh-Movement example}
\begin{forest}
for tree={s sep=10mm, inner sep=0, l=0}
[CP
    [DP,name=tgt]
    [IP
        [,phantom]
        [VP
            [DP]
            [V$'$ [V] [\sout{DP},name=src]]
        ]
    ]
]
\draw[$->$] (src) to[out=south west,in=south] (tgt);
\end{forest}

\noindent \textit{The \texttt{phantom} node is useful for maintaining horizontal spacing without drawing a line, and \texttt{\textbackslash sout\{...\}} strikes through the copy.}

\subsection{Final Notes}
Here are some basic possible errors you might encounter when the compiler breaks dowm. :(

Check through all these
\begin{itemize}
    \item Always \textbf{close every bracket}.
    \item Start simple, then delve into complexity.
    \item Some are not included in main text, so please use comments to annotate your code for future reference.
\end{itemize}

\section {Drawing Trees with the \texttt{qtree} Package}
The \texttt{qtree} package uses a bracketed string format inside the \textbackslash Tree command. It’s great for quick, small trees.

Basic syntax:
\verb|\Tree [.S [.NP Kitty ] [.VP meows ] ]|

It will produce:
\Tree [.S [.NP Kitty ] [.VP meows ] ]
\subsection{Examples}
For \textit{Andy likes Billy}
\Tree [.S 
    [.NP [.ProperN Andy ] ] 
    [.VP 
        [.V$'$ 
            [.V likes ] 
            [.NP [.ProperN Billy ] ] 
        ] 
    ] 
]

\subsection{Limitations}
As it cannot do the followings:
\begin{itemize}
    \item No semantic types or lambda annotations
    \item No arrows or movement lines
    \item No control over spacing or styling
    \item No node naming or referencing
\end{itemize}
This is not our main focus.
\section{Appendix: Set theory}

\begin{center} 
\begin{tabular}{lll}
\toprule
\emph{description} & \emph{command} & \emph{output}\\
\midrule
set brackets & \verb!\{1,2,3\}! & $\{1,2,3\}$\\
element of & \verb!\in! & $\in$\\
not an element of & \verb!\not\in! & $\not\in$\\
subset of & \verb!\subset! & $\subset$\\
subset of & \verb!\subseteq! & $\subseteq$\\
not a subset of & \verb!\not\subset! & $\not\subset$\\
contains & \verb!\supset! & $\supset$\\
contains & \verb!\supseteq! & $\supseteq$\\
union & \verb!\cup! & $\cup$\\
intersection & \verb!\cap! & $\cap$\\
big union & 
\verb!\bigcup_{n=1}^{10}A_n! &
$ \bigcup_{n=1}^{10}A_{n}$\\
\addlinespace
big intersection & \verb!\bigcap_{n=1}^{10}A_n! &$ \bigcap_{n=1}^{10}A_{n}$\\
empty set & \verb!\emptyset! & $\emptyset$\\
power set & \verb!\mathcal{P}! & $\mathcal{P}$\\
minimum & \verb!\min! & $\min$\\
maximum & \verb!\max! & $\max$\\
supremum & \verb!\sup! & $\sup$\\
infimum & \verb!\inf! & $\inf$\\
limit superior & \verb!\limsup! & $\limsup$\\
limit inferior & \verb!\liminf! & $\liminf$\\
closure & \verb!\overline{A}! & $\overline{A}$\\
\bottomrule
\end{tabular}
\end{center}

\end{spacing}
\begin{center}
    \Huge
    \textbf{Thanks! Have fun with coding!}
\end{center}

\end{document}
